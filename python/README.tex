\documentclass{article}
\usepackage[utf8]{inputenc}
\usepackage{tabto}
\usepackage{hyperref}
\title{How To}
\date{November 2022}

\begin{document}

\maketitle

\section{Execution}
Toute l'exécution de ce projet repose sur un makefile qui automatise toutes les exécutions et tous les nettoyages des fichiers crees, le fichier \textbf{'Makefile'} est dans le  répertoire racine du code python du projet \textbf{'Python'}, donc pour exécuter une section du make vous devez installer \textbf{'make'} sur votre Système.

\subsection{Intaller make}
\subsubsection{Sous Linux}

\tabto{1cm}\$ sudo apt install make
    
\subsubsection{Sous Mac}

\tabto{1cm}\$ brew install make

\subsubsection{Sous Windows}

\tabto{1cm}\$ Suivez les instructions sur ce site \href{https://www.gnu.org/software/make/}{gnu.org}

\section{Execution des tests}
\subsection{Tester les fonctions d'outils}
\subsubsection{Les fonctions d'outils}
\tabto{1cm} - decomposition
\tabto{1cm} - taille\_arbre\_compresse\_robdd
\tabto{1cm} - completion
\tabto{1cm} - table
\tabto{1cm} - hash\_luka

\subsubsection{Les fonctions d'outils testees}
\tabto{1cm} - decomposition
\tabto{1cm} - taille\_arbre\_compresse\_robdd
\tabto{1cm} - completion
\tabto{1cm} - table

\subsubsection{Commande}
\tabto{1cm}\$ make test\_tools

\subsection{Tester les fonctions Tree}
\subsubsection{Les fonctions Tree}
\tabto{1cm} - makeLeaf
\tabto{1cm} - makeNode
\tabto{1cm} - cons\_arbre


\subsubsection{Les fonctions Tree testees}
\tabto{1cm} - makeLeaf
\tabto{1cm} - makeNode
\tabto{1cm} - cons\_arbre

\subsubsection{Commande}
\tabto{1cm}\$ make test\_tree

\section{Execution du Main}

\subsection{Commande}
\tabto{1cm}\$ make exec\_main

\subsection{Parametres d'entree et les fichiers de sortie}
\subsubsection{Parametres d'entree}
\tabto{1cm}- Un nombre qui représente une fonction booléenne par sa représentation binaire. Ce nombre sera demandé par le processus et sera entré par l'invite de commande.

\subsubsection{Fichiers de sortie}
\tabto{2cm} Tous les fichiers générés sont des \textit{.dot} qui permettent de dessiner les arbres, pour pouvoir le faire, il faut mettre ces fichiers sur le site \href{https://graphs.grevian.org/graph}{graphs.grevian.org}.Ces fichiers sont dans le répertoire \textbf{./dot}: 

\tabto{1cm} -  compression\_arbre : qui dessine l'arbre compressé par la première méthode de compression
\tabto{1cm} -  compression\_bdd : qui dessine l'arbre compressé par la deuxième méthode de compression qui est celle de ROBDD
\tabto{1cm} -  cost\_arbre : qui dessine l'arbre non compressé de la fonction 
\tabto{1cm} -  luka\_arbre : qui dessine l'arbre étiqueté par les mots Lukasiewicz non compressé de la fonction




\end{document}
\documentclass{article}
\setlength{\oddsidemargin}{10pt} 
\setlength{\evensidemargin}{10pt}
\setlength{\textwidth}{481pt}
\usepackage[utf8]{inputenc}
\usepackage{tabto}
\usepackage{hyperref}
\title{How To}
\date{November 2022}

\begin{document}

\maketitle

\section{Execution}
- Toute l'exécution de ce projet repose sur un makefile qui automatise toutes les exécutions et tous les nettoyages des fichiers crees, le fichier \textbf{'Makefile'} est dans le  répertoire racine du code python du projet \textbf{'Python'}, donc pour exécuter une section du make vous devez installer \textbf{'make'} sur votre Système.

- Pour les experimentation, on utilise \textbf{gnuplot} pour dessiner les graphes.
\subsection{Intaller make}
\subsubsection{Sous Linux}

\tabto{1cm}\$ sudo apt install make
    
\subsubsection{Sous Mac}

\tabto{1cm}\$ brew install make

\subsubsection{Sous Windows}

\tabto{1cm}\$ Suivez les instructions sur ce site \href{https://www.gnu.org/software/make/}{gnu.org}

\subsection{Intaller gnuplot}
\subsubsection{Sous Linux}

\tabto{1cm}\$ sudo apt-get install gnuplot
    
\subsubsection{Sous Mac}

\tabto{1cm}\$ sudo port install gnuplot

\subsubsection{Sous Windows}

\tabto{1cm}\$ Suivez les instructions sur ce site \href{http://www.astrosurf.com/buil/isis/lisa_start/howto_gnuplot.htm}{www.astrosurf.com}

\section{Execution des tests}
\subsection{Tester les fonctions d'outils}
\subsubsection{Les fonctions d'outils}
\tabto{1cm} - decomposition
\tabto{1cm} - taille\_arbre\_compresse\_robdd
\tabto{1cm} - completion
\tabto{1cm} - table
\tabto{1cm} - hash\_luka

\subsubsection{Les fonctions d'outils testees}
\tabto{1cm} - decomposition
\tabto{1cm} - taille\_arbre\_compresse\_robdd
\tabto{1cm} - completion
\tabto{1cm} - table

\subsubsection{Commande}
\tabto{1cm}\$ make test\_tools

\subsection{Tester les fonctions Tree}
\subsubsection{Les fonctions Tree}
\tabto{1cm} - makeLeaf
\tabto{1cm} - makeNode
\tabto{1cm} - cons\_arbre


\subsubsection{Les fonctions Tree testees}
\tabto{1cm} - makeLeaf
\tabto{1cm} - makeNode
\tabto{1cm} - cons\_arbre

\subsubsection{Commande}
\tabto{1cm}\$ make test\_tree

\section{Execution du Main}

\subsection{Commande}
\tabto{1cm}\$ make exec\_main

\subsection{Parametres d'entree et les fichiers de sortie}
\subsubsection{Parametres d'entree}
\tabto{1cm}- Un nombre qui représente une fonction booléenne par sa représentation binaire. Ce nombre sera demandé par le processus et sera entré par l'invite de commande.

\subsubsection{Fichiers de sortie}
\tabto{2cm} Tous les fichiers générés sont des \textit{.dot} qui permettent de dessiner les arbres, pour pouvoir le faire, il faut mettre ces fichiers sur le site \href{https://graphs.grevian.org/graph}{graphs.grevian.org}.Ces fichiers sont dans le répertoire \textbf{./dot}: 

\tabto{1cm} -  compression\_arbre : qui dessine l'arbre compressé par la première méthode de compression
\tabto{1cm} -  compression\_bdd : qui dessine l'arbre compressé par la deuxième méthode de compression qui est celle de ROBDD
\tabto{1cm} -  cost\_arbre : qui dessine l'arbre non compressé de la fonction 
\tabto{1cm} -  luka\_arbre : qui dessine l'arbre étiqueté par les mots Lukasiewicz non compressé de la fonction


\section{Execution des experimentation}

\subsection{Commande}
\tabto{1cm}\$ make exec\_experimentation

\subsection{Parametres d'entree et les fichiers de sortie}
\subsubsection{Parametres d'entree}
\tabto{1cm}- Un nombre qui représente le nombre de variables des fonctions booléennes. Ce nombre sera demandé par le processus et sera entré par l'invite de commande.

\tabto{1cm}- le nombre  de fonctions booléennes a traite. Ce nombre sera demandé par le processus et sera entré par l'invite de commande.

\subsubsection{Fichiers de sortie}

\tabto{1cm} -  nombre\_var\{\textit{nombre de variable}\}.dat : represente les donnees de l'experimentation, il est dans le répertoire \textbf{./experimentation/gnuplot\_data}.

\tabto{1cm} -  nombre\_var\{\textit{nombre de variable}\}.png : represente les graphes de l'experimentation, il est dans le répertoire \textbf{./experimentation/gnuplot\_diagrammes}.

\section{Execution des experimentations des variables de 5 a 10}

\subsection{Commande}https://www.overleaf.com/project/63872b8612cf0cec86587f8f
\tabto{1cm}\$ make exec\_experimentations\_var\_10

\subsection{Parametres d'entree et les fichiers de sortie}
\subsubsection{Parametres d'entree}
\tabto{1cm}Aucun parametres.

\subsubsection{Fichiers de sortie}

\tabto{1cm} -  nombre\_var\{\textit{nombre de variable}\}.dat : represente les donnees de l'experimentation, il est dans le répertoire \textbf{./experimentation/gnuplot\_data}.

\tabto{1cm} -  nombre\_var\{\textit{nombre de variable}\}.png : represente les graphes de l'experimentation, il est dans le répertoire \textbf{./experimentation/gnuplot\_diagrammes}.

\tabto{1cm} -  tab.md : represente le tableau de la figure 10 de l'artice \href{https://www-apr.lip6.fr/~genitrini/doc_ens/A_Theoretical_and_Numerical_Analysis_of_the_Worst.pdf}{A\_Theoretical\_and\_Numerical\_Analysis\_of\_the\_Worst}, il est dans le répertoire \textbf{./experimentation/}.


\section{Nettoyage}
\subsection{Nettoyage des .dot}
\$ make clean\_dot
\subsection{Nettoyage des donnees et diagrammes gnuplot}
\$ make clean\_gnuplot
\subsection{Nettoyage de la table des variables}
\$ make clean\_tab
\subsection{Nettoyage des executables python}
\$ make clean
\subsection{Nettoyage du projet}
\$ make clean\_all

\end{document}
